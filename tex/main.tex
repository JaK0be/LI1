\documentclass[a4paper]{article}

\usepackage[utf8]{inputenc}
\usepackage[portuges]{babel}
\usepackage{a4wide}

\title{Projeto de Laboratórios de Informática 1\\Grupo 81}
\author{Pedro Gonçalves (A82313) \and Rodolfo Vieira (A81716)}
\date{\today}

\begin{document}

\maketitle
\newpage
\begin{abstract}
Neste relatório encontra-se descrito o processo de criação do jogo Bomberman, utilizando a linguagem de programação funcional Haskell, desde a parte da criação do jogo em si, até à parte da criação da interface gráfica e da programação de um Bot.
\end{abstract}

\tableofcontents
\newpage
\section{Introdução}
\label{sec:intro}
No âmbito da disciplina de Laboratórios de Informática 1 (LI1), foi-nos proposto a realização de um projeto que consiste na recriação, através da linguagem de programação funcional Haskell, do clássico das consolas Bomberman.
Aquando da apresentação do enunciado do projeto, o grupo sentiu-se um pouco assustado, pelo simples facto de não ter experiência em programação, mas sentiu-se igualmente motivado particularmente por dois motivos:
\begin{enumerate}
\item A possibilidade de melhoramento das nossas capacidades como programadores;
\item O facto de que no final do projeto seríamos capazes de disfrutar do fruto do nosso trabalho.
\end{enumerate}
\newpage
\section{Descrição do Problema}
\label{sec:problema}

Os problemas apresentados eram:
 \begin{enumerate}
 \item A criação de um mapa, gerado a partir de uma semente e de uma dimensão. Nesse mesmo mapa deveriam estar listados os \emph{Power-Ups} (!,+), as \emph{Pedras} (\#), os \emph{Tijolos} (?), tijolos esses que podem ser destruídos, e também os \emph{Espaços Vazios} ( ), nos quais o jogador se pode movimentar.

 \item O movimento do jogador, que envolvia as 5 diferentes ações que este poderia realizar.
  \begin{enumerate}
  \item Mover-se para cima (U);
  \item Mover-se para baixo (D);
  \item Mover-se para a direita (R);
  \item Mover-se para a esquerda (L);
  \item Colocar uma Bomba (B).
  \end{enumerate}
  Era também pedido que fosse criado um sistema que ao analizar o mapa, diria se era possível executar um determinado comando.
  
 \item O processo de guardar e continuar um jogo já começado previamente, através da implementação de um sistema de compressão/descompressão desse mesmo estado de jogo ou seja, ao comprimir, retirar certos caractéres que não são vitais ao Estado de Jogo de maneira a que o ficheiro que contenha o Estado de Jogo ocupe um menor espaço (tenha menos caractéres) que o original, em quanto que ao descomprimir, os caractéres previamente retirados são inseridos de novo no Estado de Jogo.
 
 \item A passagem do tempo, que consiste em dois processos:
 \begin{enumerate}
 \item Efeito da passagem de tempo no Estado de Jogo, ou seja programar o que acontecia com o passar de tempo (Explosão de Bombas, eliminações de Jogadores e de Power-Ups e destruição de Blocos);
 \item Criação de uma \emph{Espiral} que elimina do Estado de Jogo tudo o que esteja no seu caminho.
 \end{enumerate}

 \item A parte gráfica, que consiste na criação de uma interface gráfica de todo os problemas enumerados anteriormente. 
 
 Esta interface permite que o jogo seja jogado, pois representa graficamente todas as possibilidades: (Movimentos dos Jogadores, Colocação de Bombas, Passagem de Tempo, Explosões das Bombas e também a criação da Espiral).
 
 \item O Bot, consiste na criação de uma estratégia de jogo onde através da análise de um Estado de Jogo, é executado um determinado comando, tendo como objetivo a criação de um Jogador autónomo que consiga lutar contra outros Jogadores sem a interferência de um utilizador. 

\end{enumerate}
\newpage
\section{Concepção da Solução}
\label{sec:solucao}

Esta secção descreve o trabalho desenvolvido com vista a resolveros  problema apresentados na secção anterior.

\subsection{Estruturas de Dados}

As principais Estruturas de Dados utilizadas durante este a elaboração deste projeto foram:
\begin{enumerate}
\item Para representar o Estado de Jogo foram utilizadas [Strings];
\item Para representar o Estado, na execução da Tarefa 5,definiu-se estado como

((Float,Float),Float,[String],[Picture],Float)
\end{enumerate}

\subsection{Implementação}

As soluções encontradas aos problemas a que fomos sujeitos foram:
\begin{enumerate}
 \item Para a criação do mapa foi usado um gerador aleatório, que através do fornecimento de uma semente gera um número de 0 a 99, correspondendo cada número a um certo tipo de bloco.
 \begin{enumerate}
 \item Se o número pertencer de 0 a 1, então é representado um Power-Up Bombs atrás de um Tijolo;
 \item Se o número pertencer de 2 a 3, então é representado um Power-Up Flames atrás de um Tijolo;
 \item Se o número pertencer de 4 a 39, então é representado um Tijolo;
 \item Se o número pertencer de 40 a 99, então é representado um Espaço Vazio.
 \end{enumerate}

De seguida foi usada uma função que através do fornecimento da dimensão do mapa, gera um  mapa que consiste de blocos com a sua posição já predifinida em qualquer tipo de mapa. Sendo depois os caracteres obtidos no gerador integrados na carcaça do mapa, obtendo-se assim um mapa completo, onde so falta a listagem dos power ups, que por sua vez são originários do gerador aleatório previamente mencionado, que vai passar por um conjunto de diversas funções que decifram a sua posição relativa no mapa, e vão ordenalos numa lista que é mais tarde integrada no mapa ja quase completo, obtendo-se assim o produto final.

 \item No movimento do Jogador, foi utilizado um conjunto diverso de funções que, após o fornecimento de um \emph{Estado de Jogo}, de um \emph{Comando} que se pretende que seja efetuado e do \emph{Número identificativo do Jogador}, analisar a posição do Jogador no Estado de Jogo, verificando depois a partir do Comando fornecido qual o objeto que se encontra na posição para a qual um Jogador se pretende mover. Se esse objeto for uma \emph{Pedra} (\#), ou um \emph{Tijolo} (?), o Jogador é impedido de executar essa ação, mas se no entanto esse objeto se tratar de um \emph{Espaço Vazio} ( ), a acão que o Jogador pretendia relaizar é executada.
 
 Se o Comando que se pretenda executar for a colocação de uma Bomba, um conjunto de funções vão ser executadas com os seguintes objetivos:
 \begin{enumerate}
 \item Verifica se um dado jogador pode colocar Bombas, com base em certos fatores:
 \begin{enumerate}
 \item Se o número de Bombas que um determinado Jogador possuí for igual ou superior ao Número de \emph{Power-Ups Bomba} (+), ou se já existir uma Bomba na posição onde se pretende colocar a  nossa Bomba,o Jogador não vai ser capaz de colocar uma Bomba.
 \item Se o Jogador puder colocar Bombas,vai ser verificado o Número de \emph{Power-Ups Flame} (!) que o Jogador possui, com o objetivo de determinar o raio da Bomba que será colocada, sendo depois criada uma \emph{String} que possui as informações sobre a Bomba: (Posição onde foi colocada; Identificador do Jogador que a colocou; Raio de Explosão; Tempo até à Explosão da Bomba), \emph{String} essa que vai ser adicionada à \emph{[String]} do Estado de Jogo, logo após das \emph{String} com as informações sobre os \emph{Power-Ups}.
 \end{enumerate} 
 \end{enumerate}

 \item Na Compressão do Estado de Jogo, é utilizada uma função cujo objetivo é eliminar os blocos pré-definidos do Estado de Jogo (criados na Tarefa 1), diminuindo assim o número de caractéres utilizados e consequentemente o tamanho do Estado de Jogo.
 
 Na Descompressão, é utilizada uma função que adiciona os blocos pré-definidos novamente ao Estado de Jogo. 

 \item Na passagem do tempo, é fornecida à funçâo principal um Estado de Jogo e também o tempo restante até ao final do jogo. De seguida, esta função analisa o Estado de Jogo e faz com que ocorra a passagem de uma Unidade de Tempo. Caso esta Passagem de Tempo faça com que uma bomba exploda, a função analisa as alterações que a explosão vai gerar no Estado de Jogo, ou seja, vai analisar todos os objetos que se encontram dentro do raio de explosão da bomba e vai verificar se estes serão destruidos ou não:
 \begin{enumerate}
 \item Se o objeto for uma Pedra, a função faz com que o Raio de Explosão da Bomba não ultrapasse a Pedra e faz também com que a Pedra não seja destruída;
 \item Se o objeto for um Espaço Vazio a função vai fazer com que o raio da Bomba continue a propagar-se;
 \item Se o objeto for um Tijolo a função vai destruir esse mesmo Tijolo e substitui-o por um Espaço Vazio, parando assim a propagação da Bomba;
 \item Se num dos Espaços Vazios em cima mencionados, a Bomba se depare com um Power-Up "destapado", o seu Raio de Explosão pára de se propagar, e consequentemente elimina esse mesmo Power-Up do Estado de Jogo;
 \item Se exista um Jogador num dos Espaços Vazios afetados pelo raio da bomba, este será eliminado do Estado de Jogo, mas neste caso o raio da Bomba continua a propagar-se;
 \item Se num dos espaços vazios exista uma Bomba, o tempo até à Explosão desta, vai ser reduzido para 1 Unidade de Tempo.
\end{enumerate}
 \item Na criação da Interface Gráfica, são utilizadas diversas funções com o objetivo de representar gráficamente um determinado Estado (O tipo Estado é definido previamente):
 \begin{enumerate}
 \item Uma função que desenha um Estado de Jogo fazendo o Translate de certas imagens para determinadas posições, conforme as informações fornecidas pelo tipo \emph{Estado} (Dimensão e [String] com informações sobre o Estado de Jogo);
 \item Uma função que faz com que ocorra a passagem de tempo, permitindo assim que ocorram todos os fenómenos que dependam da Passagem de Tempo;
 \item Uma função que define o Frame Rate do Jogo;
 \item E por fim uma função que permite que o Jogador execute Comandos, e que provoque as alterações do Estado de Jogo ocorridas devido à execução desse Comando.
 \end{enumerate}

 \item Na criação do Bot, um conjunto diverso de funções foi criado com o objetivo de analisar o Estado de Jogo. O Bot está programado para perseguir o Tijolo mais próximo e quando chega perto deste, coloca uma Bomba para o destruir, repetindo este processo múltiplas vezes, mas antes de cada movimento ser executado , é analisado o lugar para o qual o bot se pretende mover:
 \begin{enumerate}
 \item Se a posição para a qual o Bot se pretende mover estiver no Raio de uma Bomba, o movimento é cancelado e o Bot permanece na mesma posição (Se antes do movimento ser executado o bot já se encontrar no Raio de Explosão de uma Bomba, o Bot irá deslocar-se para uma posição onde não seja afetado por essa mesma Bomba);
 \item Se na posição para a qual o Bot se pretenda mover exista uma pedra, o Bot irá mudar de estratégia com vista a não se tentar deslocar para uma posição que contenha uma pedra.
 \item Se a Espiral já se estiver a formar, o Bot adota uma estratégia defensiva, cujo principal objetivo é dirigir-se para o centro do mapa de maneira a ser o último Bot a morrer (Neste processo, continuam a ser executadas as funções em cima anunciadas). 
 
 \end{enumerate}
\end{enumerate}

\subsection{Testes}

\begin{enumerate}

\item (["\#\#\#\#\#\#\#\#\#","\#       \#","\# \#?\# \# \#","\#     ? \#","\#?\# \# \#?\#","\# ?  ?  \#","\# \#?\#?\# \#","\#  ??   \#","\#\#\#\#\#\#\#\#\#"]) \par 
    O teste acima representado foi utilizado em funções que apenas necessitavam do mapa em si e não das informações adicionais, tendo sido mais utilizada na criação do Bot, pois a criação precisava constantemente de ser testada, para ver se o Bot reagia da maneira que seria esperado.

\item (["\#\#\#\#\#\#\#\#\#","\#       \#","\# \#?\# \# \#","\#     ? \#","\#?\# \# \#?\#","\# ?  ?  \#","\# \#?\#?\# \#","\#  ??   \#","\#\#\#\#\#\#\#\#\#","+ 3 3","! 5 5","* 7 7 1 1 10","0 4 3 +","1 7 7"]) \par
 Este teste foi o mais utilizad pois tem uma dimensão favorável a alterações, sendo muito útil para testar cenários diferentes mais facilmente.
\item (["\#\#\#\#\#\#\#\#\#\#\#\#\#\#\#","\#  ? ??  ?    \#","\# \#?\# \#?\#?\# \# \#","\# ?  ?? ??  ? \#","\# \#?\# \# \#?\#?\# \#","\#? ?  ???  ? ?\#","\#?\#?\#?\#?\# \#?\#?\#","\#   ? ????   ?\#","\# \# \# \# \#?\#?\# \#","\#  ?? ?     ??\#","\# \# \# \# \#?\#?\# \#","\#?  ??  ?     \#","\# \#?\#?\# \#?\# \# \#","\#  ? ????? ?  \#","\#\#\#\#\#\#\#\#\#\#\#\#\#\#\#","+ 8 3","+ 12 3","+ 7 6","+ 12 9","+ 5 12","! 3 5","! 13 6","! 7 7","* 1 2 0 2 10","* 13 2 1 2 10","* 7 3 0 2 10","* 1 4 2 2 10","* 13 8 1 2 10","* 10 13 0 2 10","0 9 6 +++ !","1 5 2 ++ !","2 1 2 !"]) \par 
Por fim, utilizamos este teste de dimensão 15, com uma complexidade muito superior relativamente aos outros testes previamente utilizados, sendo maioritariamente usado para testar a compressão/descompressão na tarefa3 (Devido à sua elevada complexidade, permite-nos testar muitas mais possibilidades numa só aplicação da função, o que nos permite encontrar possíveis erros nas Tarefas mais facilmente.

\end{enumerate} 
\newpage
\section{Conclusões}
\label{sec:conclusao}

Com a realização deste trabalho e após a execução de todas as Tarefas propostas no enunciado do projeto com vista à reprodução do jogo Bomberman, consideramos que apesar das adversidades pelas quais tivemos que percorrer, conseguimos melhorar as nossas capacidades de programador e que este trabalho foi uma mais valia para o nosso futuro profissional. 

\end{document}
